% !tex encoding = utf-8 unicode

% this is a simple template for a latex document using the "article" class.
% see "book", "report", "letter" for other types of document.

\documentclass[11pt]{article} % use larger type; default would be 10pt



%%%%%%%%%%%%%%%%%%%%%%%%%%%%%%%%%%%%%%%%%%%%%%%%%%%%%%%%%%%%%%%%%%%%%%%%%%%%%%%%%%%%%%%%%%%%%%%%%%%%%%%%%%%%%%%%%%%%%%%%%%%%%%%%%%%%%%%%%%%%%%%%%%%%%%%%%%%%%%%%%%%%%%%%%%%%%%%%
%% packages
\usepackage[
    type={CC},
    modifier={by-nc-sa},
    version={3.0},
]{doclicense}

\usepackage[utf8]{inputenc} % set input encoding (not needed with xelatex)
\usepackage[strict]{changepage}
    
    %%% examples of article customizations
    % these packages are optional, depending whether you want the features they provide.
    % see the latex companion or other references for full information.
    
    
    %%% page dimensions
\usepackage{geometry} % to change the page dimensions
\geometry{a4paper} % or letterpaper (us) or a5paper or....
    % \geometry{margin=2in} % for example, change the margins to 2 inches all round
    % \geometry{landscape} % set up the page for landscape
    %   read geometry.pdf for detailed page layout information
 \usepackage{numdef}
   
\usepackage{graphicx} % support the \includegraphics command and options
    % some of the article customisations are relevant for this class
\usepackage{amsmath,amsthm}
\usepackage{amsfonts} % math fonts such as \mathbb{}
\usepackage{amssymb} % \therefore
% \usepackage{bickham}
\usepackage{hyperref}
\usepackage{cryptocode}
\usepackage{framed} 
    % \usepackage[parfill]{parskip} % activate to begin paragraphs with an empty line rather than an indent
    
    %%% packages
\usepackage{booktabs} % for much better looking tables
\usepackage{array} % for better arrays (eg matrices) in maths
\usepackage{paralist} % very flexible & customisable lists (eg. Enumerate/itemize, etc.)
\usepackage{verbatim} % adds environment for commenting out blocks of text & for better verbatim
\usepackage{subfig} % make it possible to include more than one captioned figure/table in a single float
    % % these packages are all incorporated in the memoir class to one degree or another...
\usepackage{mathrsfs}
\usepackage{booktabs}
\usepackage{makecell}
\usepackage{adjustbox}

\usepackage{pgfplots}

\renewcommand\theadalign{bc}
\renewcommand\theadfont{\bfseries}
\renewcommand\theadgape{\Gape[4pt]}
  
    %%% headers & footers
\usepackage{fancyhdr} % this should be set after setting up the page geometry
\pagestyle{fancy} % options: empty , plain , fancy
\renewcommand{\headrulewidth}{0pt} % customise the layout...
\lhead{}\chead{}\rhead{}
\lfoot{}\cfoot{\thepage}\rfoot{}
    
    
    %%% section title appearance
\usepackage{sectsty}
\allsectionsfont{\sffamily\mdseries\upshape} % (see the fntguide.pdf for font help)
    % (this matches context defaults)
    
    %%% toc (table of contents) appearance
\usepackage[nottoc,notlof,notlot]{tocbibind} % put the bibliography in the toc
\usepackage[titles,subfigure]{tocloft} % alter the style of the table of contents
\renewcommand{\cftsecfont}{\rmfamily\mdseries\upshape}
\renewcommand{\cftsecpagefont}{\rmfamily\mdseries\upshape} % no bold!

 %%% end article customizations
 
 %%%%%%%%%%%%%%%%%%%%%%%%%%%%%%%%%%%%%%%%%%%%%%%%%%%%%%%%%%%%%%%%%%%%%%%%%%%%%%%%%%%%%%%%%%%%%%%%%%%%%%%%%%%%%%%%%%%%%%%%%%%%%%%%%%%%%%%%%%%%%%%%%%%%%%%%%%%%%%%%%%%%%%%%%%%%%%%%
 % macros
 \newcommand{\code}[1]{\texttt{#1}}
\newcommand\tstrut{\rule{0pt}{2.6ex}}         % = `top' strut
\newcommand\bstrut{\rule[-0.9ex]{0pt}{0pt}}   % = `bottom' strut
\newcommand{\bgamma}{\boldsymbol{\gamma}}
\newcommand{\bsigma}{\boldsymbol{\sigma}}
\newcommand{\plonk}{\ensuremath{\mathcal{P} \mathfrak{lon}\mathcal{K}}\xspace}
\newcommand{\cq}{\ensuremath{\mathpgoth{cq} }\xspace}
\newcommand{\flookup}{\ensuremath{\mathsf{\mathpgoth{Flookup}}}\xspace}
\newcommand{\caulkp}{\ensuremath{\mathsf{\mathrel{Caulk}\mathrel{\scriptstyle{+}}}}\xspace}
\newcommand{\caulk}{\ensuremath{\mathsf{Caulk}}\xspace}
\newcommand{\plookup}{\ensuremath{\mathpgoth{plookup}}\xspace}
\newcommand{\tablegroup}{\ensuremath{\mathbb{H}}\xspace}
\newcommand{\V}{\ensuremath{\mathbf{V} }\xspace}



% \newcommand{\plonk}{\ensuremath{\mathtt{PLONK}}\xspace}
\newcommand{\papertitle}{\cq:\footnote{Pronounced ``seek you''.} Cached quotients for fast lookups}
%\newcommand{\authorname}}
\newcommand{\company}{}
\title{ \bf \papertitle \\[0.72cm]}
\author{ Liam Eagen \and  Dario Fiore \\ \tt{IMDEA}  \and Ariel Gabizon \\ \tt{Zeta Function Technologies} } 
% 
% 	\large{\authorname} \\[0.5cm] \large{\company}
% 	\\ {DRAFT}
%}
%\date{} % activate to display a given date or no date (if empty),

% otherwise the current date is printed 
\DeclareMathAlphabet{\mathpgoth}{OT1}{pgoth}{m}{n}	
\ProvidesPackage{numdef}



%% Ariel Macros:
\num\newcommand{\G1}{\ensuremath{{\mathbb G}_1}\xspace}
\newcommand{\Gi}{\ensuremath{{\mathbb G}_i}\xspace}
%\newcommand{\G}{\ensuremath{{\mathbb G}}\xspace}
\newcommand{\Gstar}{\ensuremath{{\mathbb G}^*}\xspace}
\newcommand{\x}{\ensuremath{\mathbf{x}}\xspace}
\newcommand{\z}{\ensuremath{\mathbf{z}}\xspace}
\newcommand{\X}{\ensuremath{\mathbf{X}}\xspace}

\num\newcommand{\G2}{\ensuremath{{\mathbb G}_2}\xspace}
%\num\newcommand{\G11}{\ensuremath{\G1\setminus \set{0} }\xspace}
%\num\newcommand{\G21}{\ensuremath{\G2\setminus \set{0} }\xspace}
\newcommand{\grouppair}{\ensuremath{G^*}\xspace}
\newcommand{\prvperm}{\ensuremath{\mathrm{P_{\mathsf{\sigma}}}}\xspace}
\newcommand{\verperm}{\ensuremath{\mathrm{V_{\mathsf{\sigma}}}}\xspace}
\newcommand{\alg}{\ensuremath{\mathscr{A}}\xspace}

\newcommand{\Gt}{\ensuremath{{\mathbb G}_t}\xspace}
\newcommand{\F}{\ensuremath{\mathbb F}\xspace}
\newcommand{\Fstar}{\ensuremath{\mathbb F^*}\xspace}

\newcommand{\help}[1]{$#1$-helper\xspace}
\newcommand{\randompair}[1]{\ensuremath{\mathsf{randomPair}(#1)}\xspace}
\newcommand{\pair}[1]{$#1$-pair\xspace}
\newcommand{\pairs}[1]{$#1$-pairs\xspace}
\newcommand{\chalpoint}{\ensuremath{\mathfrak{z}}\xspace}

\newcommand{\pairone}[1]{\G1-$#1$-pair\xspace}
\newcommand{\pairtwo}[1]{\G2-$#1$-pair\xspace}
\newcommand{\sameratio}[2]{\ensuremath{\mathsf{SameRatio}(#1,#2)}\xspace}
\newcommand{\vecc}[2]{\ensuremath{(#1)_{#2}}\xspace}
\newcommand{\players}{\ensuremath{[n]}\xspace}
\newcommand{\adv}{\ensuremath{\mathcal A}\xspace}
\newcommand{\advprime}{\ensuremath{{\mathcal A}'}\xspace}
\newcommand{\extprime}{\ensuremath{E'}\xspace}
\newcommand{\advrand}{\ensuremath{\mathsf{rand}_{\adv}}\xspace}
\num\newcommand{\srs1}{\ensuremath{\mathsf{srs_1}}\xspace}
\num\newcommand{\srs2}{\ensuremath{\mathsf{srs_2}}\xspace}
\num\newcommand{\srs0}{\ensuremath{\mathsf{srs}}\xspace}
\newcommand{\srsi}{\ensuremath{\mathsf{srs_i}}\xspace}
\newcommand{\com}{\ensuremath{\mathsf{com}}\xspace}
\newcommand{\comperm}{\ensuremath{\mathsf{com_{\sigma}}}\xspace}
\newcommand{\cm}{\ensuremath{\mathsf{cm}}\xspace}
\newcommand{\cmsig}{\ensuremath{\mathsf{cm_\sigma}}\xspace}
\newcommand{\open}{\ensuremath{\mathsf{open}}\xspace}
\newcommand{\openperm}{\ensuremath{\mathsf{open_{\sigma}}}\xspace}
\newcommand{\sigof}[1]{\ensuremath{\sigma(#1)}\xspace}
\newcommand{\proverexp}{\ensuremath{\mathsf{e}}\xspace}
\newcommand{\reducedelems}{\ensuremath{\mathsf{r}}\xspace}

\newcommand{\ci}{\ensuremath{\mathrm{CI}}\xspace}
\renewcommand{\deg}{\ensuremath{\mathrm{deg}}\xspace}
\newcommand{\pairvec}[1]{$#1$-vector\xspace}
\newcommand{\Fq}{\ensuremath{\mathbb{F}_q}\xspace}
\newcommand{\randpair}[1]{\ensuremath{\mathsf{rp}_{#1}}\xspace}
\newcommand{\randpairone}[1]{\ensuremath{\mathsf{rp}_{#1}^{1}}\xspace}
\newcommand{\abase}{\ensuremath{A_{\mathrm{\mathbf{0}}}}\xspace}
\newcommand{\bbase}{\ensuremath{B_{\mathrm{\mathbf{0}}}}\xspace}
\newcommand{\cbase}{\ensuremath{C_{\mathrm{\mathbf{0}}}}\xspace}
\newcommand{\bliop}[1]{\ensuremath{\mathsf{#1{\text{-}}{BLIOP}}}\xspace}
\newcommand{\blop}{\ensuremath{\mathsf{\mathscr{BL} {\text{-}}IOP}}\xspace}
\newcommand{\openprotinput}{\ensuremath{\mathcal{O}}\xspace}
\newcommand{\amid}{\ensuremath{A_{\mathrm{mid}}}\xspace}
\newcommand{\bmid}{\ensuremath{B_{\mathrm{mid}}}\xspace}
\newcommand{\cmid}{\ensuremath{C_{\mathrm{mid}}}\xspace}

\newcommand{\negl}{\ensuremath{\mathsf{negl}(\lambda)}\xspace}
\newcommand{\randpairtwo}[1]{\ensuremath{\mathsf{rp_{#1}^2}}\xspace}%the randpair in G2
\newcommand{\nilp}{\ensuremath{\mathscr N}\xspace}
\newcommand{\groupgen}{\ensuremath{\mathscr G}\xspace}
\newcommand{\qap}{\ensuremath{\mathscr Q}\xspace}
\newcommand{\polprot}[4]{$(#1,#2,#3,#4)$-polynomial protocol}
\newcommand{\rangedprot}[5]{$#5$-ranged $(#1,#2,#3,#4)$-polynomial protocol}

\newcommand{\rej}{\ensuremath{\mathsf{rej}}\xspace}
\newcommand{\acc}{\ensuremath{\mathsf{acc}}\xspace}
\newcommand{\res}{\ensuremath{\mathsf{res}}\xspace}
\newcommand{\sha}[1]{\ensuremath{\mathsf{COMMIT}(#1)}\xspace}
 \newcommand{\shaa}{\ensuremath{\mathsf{COMMIT}}\xspace}
 \newcommand{\comm}[1]{\ensuremath{\mathsf{comm}_{#1}}\xspace}
 \newcommand{\defeq}{:=}

\newcommand{\B}{\ensuremath{\vec{B}}\xspace}
\newcommand{\dom}{\ensuremath{H}\xspace}
\newcommand{\C}{\ensuremath{\vec{C}}\xspace}
\newcommand{\Btwo}{\ensuremath{\vec{B_2}}\xspace}
\newcommand{\treevecsimp}{\ensuremath{(\tau,\rho_A,\rho_A \rho_B,\rho_A\alpha_A,\rho_A\rho_B\alpha_B, \rho_A\rho_B\alpha_C,\beta,\beta\gamma)}\xspace}% The sets of elements used in simplifed relation tree in main text body
\newcommand{\rcptc}{random-coefficient subprotocol\xspace}
\newcommand{\rcptcparams}[2]{\ensuremath{\mathrm{RCPC}(#1,#2)}\xspace}
\newcommand{\verifyrcptcparams}[2]{\ensuremath{\mathrm{\mathsf{verify}RCPC}(#1,#2)}\xspace}
\newcommand{\randadv}{\ensuremath{\mathsf{rand}_{\adv}}\xspace}
 \num\newcommand{\ex1}[1]{\ensuremath{ #1\cdot g_1}\xspace}
 \num\newcommand{\ex2}[1]{\ensuremath{#1\cdot g_2}\xspace}
 \newcommand{\pr}{\mathrm{Pr}}
 \newcommand{\powervec}[2]{\ensuremath{(1,#1,#1^{2},\ldots,#1^{#2})}\xspace}
 \newcommand{\partition}{\ensuremath{{\mathcal T}}\xspace}
 \newcommand{\partof}[1]{\ensuremath{{\partition_{#1}}}\xspace}
\num\newcommand{\out1}[1]{\ensuremath{\ex1{\powervec{#1}{d}}}\xspace}
\num\newcommand{\out2}[1]{\ensuremath{\ex2{\powervec{#1}{d}}}\xspace}
 \newcommand{\nizk}[2]{\ensuremath{\mathrm{NIZK}(#1,#2)}\xspace}% #2 is the hash concatenation input
 \newcommand{\verifynizk}[3]{\ensuremath{\mathrm{VERIFY\mhyphen NIZK}(#1,#2,#3)}\xspace}
\newcommand{\protver}{protocol verifier\xspace} 
\newcommand{\hash}{\ensuremath{\mathcal{H}}\xspace} 
\newcommand{\mulgroup}{\ensuremath{\F^*}\xspace}
\newcommand{\lag}[1]{\ensuremath{L_{#1}}\xspace} 
\newcommand{\sett}[2]{\ensuremath{\set{#1}_{#2}}\xspace}
\newcommand{\omegaprod}{\ensuremath{\alpha_{\omega}}\xspace}
\newcommand{\lagvec}[1]{\ensuremath{\mathrm{LAG}_{#1}}\xspace}
\newcommand{\trapdoor}{\ensuremath{r}}
\newcommand{\trapdoorext}{\ensuremath{r_{\mathrm{ext}}}\xspace}
\newcommand{\trapdoorsim}{\ensuremath{r_{\mathrm{sim}}}\xspace}
\renewcommand{\mod}{\ensuremath{\;\mathrm{mod}\;}}
\newcommand{\hsub}{\ensuremath{H^*}\xspace}
\num\newcommand{\enc1}[1]{\ensuremath{\left[#1\right]_1}\xspace}
\newcommand{\enci}[1]{\ensuremath{\left[#1\right]_i}\xspace}
\num\newcommand{\enc2}[1]{\ensuremath{\left[#1\right]_2}\xspace}
\newcommand{\gen}{\ensuremath{\mathsf{gen}}\xspace}
\newcommand{\prv}{\ensuremath{\mathsf{\mathbf{P}}}\xspace}
\newcommand{\prvpoly}{\ensuremath{\mathrm{P_{\mathsf{poly}}}}\xspace}
\newcommand{\prvpc}{\ensuremath{\mathrm{P_{\mathsf{PC}}}}\xspace}
\newcommand{\verpoly}{\ensuremath{\mathrm{V_{\mathsf{poly}}}}\xspace}
\newcommand{\verpc}{\ensuremath{\mathrm{V_{\mathsf{PC}}}}\xspace}
\newcommand{\ideal}{\ensuremath{\mathcal{I}}\xspace}
\newcommand{\prf}{\ensuremath{\pi}\xspace}
\newcommand{\simprv}{\ensuremath{\mathrm{P^{sim}}}\xspace}

%\newcommand{\enc}[1]{\ensuremath{\left[#1\right]}\xspace}
%\num\newcommand{\G0}{\ensuremath{\mathbf{G}}\xspace}
\newcommand{\GG}{\ensuremath{\mathbf{G^*}}\xspace}  % would have liked to call this G01 but problem with name
\num\newcommand{\g0}{\ensuremath{\mathbf{g}}\xspace}
\newcommand{\inst}{\ensuremath{\phi}\xspace}
\newcommand{\inp}{\ensuremath{\mathsf{x}}\xspace}
\newcommand{\wit}{\ensuremath{\omega}\xspace}
\newcommand{\ver}{\ensuremath{\mathsf{\mathbf{V}}}\xspace}
\newcommand{\per}{\ensuremath{\mathsf{\mathbf{P}}}\xspace}
\newcommand{\sonic}{\ensuremath{\mathsf{Sonic}}\xspace}
\newcommand{\aurora}{\ensuremath{\mathsf{Aurora}}\xspace}
\newcommand{\auroralight}{\ensuremath{\mathsf{Auroralight}}\xspace}
\newcommand{\groth}{\ensuremath{\mathsf{Groth'16}}\xspace}
\newcommand{\kate}{\ensuremath{\mathsf{KZG}}\xspace}
\newcommand{\rel}{\ensuremath{\mathcal{R}}\xspace}
\newcommand{\lang}{\ensuremath{\mathcal{L}}\xspace}
\newcommand{\ext}{\ensuremath{E}\xspace}
\newcommand{\params}{\ensuremath{\mathsf{params}_{\inst}}\xspace}
\newcommand{\protparams}{\ensuremath{\mathsf{params}_{\inst}^\advv}\xspace}
\num\newcommand{\p1}{\ensuremath{P_1}\xspace}
\newcommand{\advv}{\ensuremath{ {\mathcal A}^{\mathbf{*}}}\xspace} % the adversary that uses protocol adversary as black box
\newcommand{\crs}{\ensuremath{\sigma}\xspace}
%\num\newcommand{\crs1}{\ensuremath{\mathrm{\sigma}_1}\xspace}
%\num\newcommand{\crs2}{\ensuremath{\mathrm{\sigma}_2}\xspace}
\newcommand{\set}[1]{\ensuremath{\left\{#1\right\}}\xspace}
\newcommand{\hgen}{\ensuremath{\mathbf{\omega}}\xspace}
\newcommand{\vgen}{\ensuremath{\mathbf{g}}\xspace}
\renewcommand{\sim}{\ensuremath{\mathsf{sim}}\xspace}%the distribution of messages when \advv simulates message of \p1
\newcommand{\real}{\ensuremath{\mathsf{real}}\xspace}%the distribution of messages when \p1 is honest and \adv controls rest of players
 \newcommand{\koevec}[2]{\ensuremath{(1,#1,\ldots,#1^{#2},\alpha,\alpha #1,\ldots,\alpha #1^{#2})}\xspace}
\newcommand{\mida}{\ensuremath{A_{\mathrm{mid}}}\xspace}
\newcommand{\midb}{\ensuremath{B_{\mathrm{mid}}}\xspace}
\newcommand{\midc}{\ensuremath{C_{\mathrm{mid}}}\xspace}
\newcommand{\chal}{\ensuremath{\mathsf{challenge}}\xspace}
\newcommand{\attackparams}{\ensuremath{\mathsf{params^{pin}}}\xspace}
\newcommand{\pk}{\ensuremath{\mathsf{pk}}\xspace}
\newcommand{\attackdist}[2]{\ensuremath{AD_{#1}}\xspace}
\renewcommand{\neg}{\ensuremath{\mathsf{negl}(\lambda)}\xspace}
\newcommand{\ro}{\ensuremath{{\mathcal R}}\xspace}
\newcommand{\elements}[1]{\ensuremath{\mathsf{elements}_{#1}}\xspace}
 \num\newcommand{\elmpowers1}[1]{\ensuremath{\mathrm{\mathsf{e}}^1_{#1}}\xspace}
 \num\newcommand{\elmpowers2}[1]{\ensuremath{\mathrm{\mathsf{e}}^2_{#1}}\xspace}
\newcommand{\elempowrs}[1]{\ensuremath{\mathsf{e}_{#1}}\xspace}
 \newcommand{\secrets}{\ensuremath{\mathsf{secrets}}\xspace}
 \newcommand{\polysofdeg}[1]{\ensuremath{\F_{< #1}[X]}\xspace}
 \newcommand{\bivar}[1]{\ensuremath{\F_{< #1}[X,Y]}\xspace}
 \newcommand{\sig}{\ensuremath{\mathscr{S}}\xspace}
 \newcommand{\prot}{\ensuremath{\mathscr{P}}\xspace}
 \newcommand{\PCscheme}{\ensuremath{\mathscr{S}}\xspace}
 \newcommand{\protprime}{\ensuremath{\mathscr{P^*}}\xspace}
 \newcommand{\sigprv}{\ensuremath{\mathsf{P_{sc}}}\xspace}
 \newcommand{\sigver}{\ensuremath{\mathsf{V_{sc}}}\xspace}
 \newcommand{\sigpoly}{\ensuremath{\mathsf{S_{\sigma}}}\xspace}
 \newcommand{\idpoly}{\ensuremath{\mathsf{S_{ID}}}\xspace}
\newcommand{\idpolyevala}{\ensuremath{\mathsf{\bar{s}_{ID1}}}\xspace}
\newcommand{\sigpolyevala}{\ensuremath{\mathsf{\bar{s}_{\sigma1}}}\xspace}
\newcommand{\sigpolyevalb}{\ensuremath{\mathsf{\bar{s}_{\sigma2}}}\xspace}
\newcommand{\bctv}{\ensuremath{\mathsf{BCTV}}\xspace}
\newcommand{\PI}{\ensuremath{\mathsf{PI}}\xspace}
\newcommand{\PIb}{\ensuremath{\mathsf{PI_B}}\xspace}
\newcommand{\PIc}{\ensuremath{\mathsf{PI_C}}\xspace}
\newcommand{\dl}[1]{\ensuremath{\widehat{#1}}\xspace}
\newcommand{\obgen}{\ensuremath{\mathcal O}\xspace}
\newcommand{\PC}{\ensuremath{\mathscr{P}}\xspace}
\newcommand{\permscheme}{\ensuremath{\sigma_\mathscr{P}}\xspace}

	
\newcommand{\selleft}{\ensuremath{\mathbf{q_L}}\xspace}
\newcommand{\selright}{\ensuremath{\mathbf{q_R}}\xspace}
\newcommand{\selout}{\ensuremath{\mathbf{q_O}}\xspace}
\newcommand{\selmult}{\ensuremath{\mathbf{q_M}}\xspace}
\newcommand{\selconst}{\ensuremath{\mathbf{q_C}}\xspace}
\newcommand{\selectors}{\ensuremath{\mathcal{Q}}\xspace}
\newcommand{\lvar}{\ensuremath{\mathbf{a}}\xspace}
\newcommand{\vars}{\ensuremath{\mathcal{V}}\xspace}
\newcommand{\rvar}{\ensuremath{\mathbf{b}}\xspace}
\newcommand{\ovar}{\ensuremath{\mathbf{c}}\xspace}
\newcommand{\pubvars}{\ensuremath{\mathcal{I}}\xspace}
\newcommand{\assignment}{\ensuremath{\mathbf{x}}\xspace}
\newcommand{\constsystem}{\ensuremath{\mathscr{C}}\xspace}
\newcommand{\relof}[1]{\ensuremath{\rel_{#1}}\xspace}
\newcommand{\pubinppoly}{\ensuremath{\mathsf{PI}}\xspace}
\newcommand{\sumi}[1]{\sum_{i\in[#1]}}
\newcommand{\summ}[1]{\sum_{i\in[#1]}}
\newcommand{\sumj}[1]{\sum_{j\in[#1]}}
\newcommand{\ZeroH}{\ensuremath{Z_{H}} \xspace}
\newcommand{\lpoly}{\ensuremath{\mathsf{a}}\xspace}
\newcommand{\rpoly}{\ensuremath{\mathsf{b}}\xspace}
\newcommand{\opoly}{\ensuremath{\mathsf{c}}\xspace}
\newcommand{\idpermpoly}{\ensuremath{\mathsf{z}}\xspace}
\newcommand{\lagrangepoly}{\ensuremath{\mathsf{L}}\xspace}
\newcommand{\zeropoly}{\ensuremath{\mathsf{\ZeroH}}\xspace}
\newcommand{\selmultpoly}{\ensuremath{\mathsf{q_M}}\xspace}
\newcommand{\selleftpoly}{\ensuremath{\mathsf{q_L}}\xspace}
\newcommand{\selrightpoly}{\ensuremath{\mathsf{q_R}}\xspace}
\newcommand{\seloutpoly}{\ensuremath{\mathsf{q_O}}\xspace}
\newcommand{\selconstpoly}{\ensuremath{\mathsf{q_C}}\xspace}
\newcommand{\idcomm}{\ensuremath{[s_{\mathsf{ID1}}]_1}\xspace}
\newcommand{\sigcomma}{\ensuremath{[s_{\mathsf{\sigma1}}]_1}\xspace}
\newcommand{\sigcommb}{\ensuremath{[s_{\mathsf{\sigma2}}]_1}\xspace}
\newcommand{\sigcommc}{\ensuremath{[s_{\mathsf{\sigma3}}]_1}\xspace}
\newcommand{\selleftcomm}{\ensuremath{[q_\mathsf{L}]_1}\xspace}
\newcommand{\selrightcomm}{\ensuremath{[q_\mathsf{R}]_1}\xspace}
\newcommand{\seloutcomm}{\ensuremath{[q_\mathsf{O}]_1}\xspace}
\newcommand{\selconstcomm}{\ensuremath{[q_\mathsf{C}]_1}\xspace}
\newcommand{\selmultcomm}{\ensuremath{[q_\mathsf{M}]_1}\xspace}

\newcommand{\multlinecomment}[1]{\directlua{-- #1}}
    

\newtheorem{lemma}{Lemma}[section]
\newtheorem{thm}[lemma]{Theorem}
\newtheorem{dfn}[lemma]{Definition}
\newtheorem{remark}[lemma]{Remark}

\newtheorem{claim}[lemma]{Claim}
\newtheorem{corollary}[lemma]{Corollary}
\newcommand{\Z}{\mathbb{Z}}
\newcommand{\R}{\mathcal{R}}
\newcommand{\crct}{\ensuremath{\mathsf{C}}\xspace}
\newcommand{\A}{\mathcal{A}}
%\newcommand{\G}{\mathcal{G}}
\newcommand{\Gr}{\mathbb{G}}
%\newcommand{\com}{\textsf{com}}  Ariel defined equivalent that also works in math mode
\newcommand{\cgen}{\text{cgen}}
\newcommand{\poly}{\ensuremath{\mathsf{poly(\lambda)}}\xspace}
\newcommand{\snark}{\ensuremath{\mathsf{snark}}\xspace}
\newcommand{\grandprod}{\mathsf{prod}}
\newcommand{\perm}{\mathsf{perm}}
%\newcommand{\open}{\mathsf{open}}
\newcommand{\update}{\mathsf{update}}
\newcommand{\Prove}{\mathcal{P}}
\newcommand{\Verify}{\mathcal{V}}
\newcommand{\Extract}{\mathcal{E}}
\newcommand{\Simulate}{\mathcal{S}}
\newcommand{\Unique}{\mathcal{U}}
\newcommand{\Rpoly}{\R{\poly}}
\newcommand{\Ppoly}{\Prove{\poly}}
\newcommand{\Vpoly}{\Verify{\poly}}
\newcommand{\Psnark}{\prv}%{\Prove{\snark}}
\newcommand{\Vsnark}{\ver}%{\Verify{\snark}}
\newcommand{\Rprod}{\R{\grandprod}}
\newcommand{\Pprod}{\Prove{\grandprod}}
\newcommand{\Vprod}{\Verify{\grandprod}}
\newcommand{\Rperm}{\R{\perm}}
\newcommand{\Pperm}{\Prove{\perm}}
\newcommand{\Vperm}{\Verify{\perm}}
% \newcommand{\zw}[1]{{\textcolor{magenta}{Zac:#1}}}
% \newcommand{\ag}[1]{{\textcolor{blue}{\emph{Ariel:#1}}}}
\newcommand{\prob}{\ensuremath{\mathrm{Pr}}\xspace}
\newcommand{\extprot}{\ensuremath{E_{\prot}}\xspace}
\newcommand{\transcript}{\ensuremath{\mathsf{transcript}}\xspace}
\newcommand{\extpc}{\ensuremath{E_{\PCscheme}}\xspace}
\newcommand{\advpc}{\ensuremath{\mathcal A_{\PCscheme}}\xspace}
\newcommand{\advprot}{\ensuremath{\mathcal A_{\prot}}\xspace}
\newcommand{\protmany}{\ensuremath{ {\prot}_k}\xspace}

\usepackage{pifont}% http://ctan.org/pkg/pifont
\newcommand{\cmark}{\ding{51}}%
\newcommand{\xmark}{\ding{55}}%
\newcommand{\marlin}{\ensuremath{\mathsf{Marlin}}\xspace}
\newcommand{\fractal}{\ensuremath{\mathsf{Fractal}}\xspace}
% \newcommand{\Rsnark}{\R{\snark}}
\newcommand{\Rsnark}{\R}
\newcommand{\subvec}[1]{\ensuremath{#1|_{\subspace}}\xspace}
\newcommand{\restricttoset}[2]{\ensuremath{#1|_{#2}}\xspace}
\newcommand{\isinvanishing}[1]{\ensuremath{\mathsf{IsInVanishing_{\subspace,#1}}}\xspace}
\newcommand{\batchedisinvanishing}[1]{\ensuremath{\mathsf{BatchedIsInVanishing_{\subspace,#1}}}\xspace}
\newcommand{\isconsistent}{\ensuremath{\mathsf{IsConsistent}}\xspace}
\newcommand{\isintable}{\ensuremath{\mathsf{IsInTable_{\subspace}}}\xspace}
\newcommand{\isinvanishingtable}[1]{\ensuremath{\mathsf{IsInVanishingTable_{\subspace,#1}}}\xspace}
\newcommand{\isvanishingsubtable}[1]{\ensuremath{\mathsf{IsVanishingSubtable_{#1}}}\xspace}
\newcommand{\haslowerdegree}{\ensuremath{\mathsf{HasLowerDegree}}\xspace}
\newcommand{\haslowdegree}[1]{\ensuremath{\mathsf{HasLowDegree_{#1}}}\xspace}
\newcommand{\issubtable}[1]{\ensuremath{\mathsf{IsSubtable_{#1}}}\xspace}
\newcommand{\isinsubtable}[2]{\ensuremath{\mathsf{IsInSubtable_{#1,#2}}}\xspace}
\newcommand{\secbasezero}[1]{\ensuremath{\hat{\tau}_{#1}}\xspace}
\newcommand{\secbase}[1]{\ensuremath{\hat{\tau}_{#1}}\xspace}
\newcommand{\secbasereg}[1]{\ensuremath{\tau_{#1}}\xspace}
\newcommand{\secset}{\ensuremath{I}\xspace}
\newcommand{\pubbase}[1]{\ensuremath{\mu_{#1}}\xspace}
\newcommand{\subspace}{\ensuremath{\mathbb{H}}\xspace}
\newcommand{\bigspace}{\ensuremath{\mathbb{V}}\xspace}
\newcommand{\subtable}{\ensuremath{T_0}\xspace}
\newcommand{\tablelang}{\ensuremath{\lang_{T}}\xspace}
\newcommand{\vanishingtablelang}{\ensuremath{\lang_{\subspace}}\xspace}
\newcommand{\nonorm}[1]{\ensuremath{\Gamma^T_{#1}}\xspace}
\newcommand{\unnorm}[2]{\ensuremath{\Gamma^{#1}_{#2}}\xspace}
\newcommand{\bigspacebase}{\ensuremath{\lambda}\xspace}
\newcommand{\bigspacegen}{\ensuremath{\mathsf{h}}\xspace}
\newcommand{\modvan}[1]{\ensuremath{\mathrm{mod\;}Z_{#1}}\xspace}
\newcommand{\extractevaltable}{\ensuremath{\mathsf{ExtractEvalTable}_{C,\tablegroup}}\xspace}
\newcommand{\witsize}{\ensuremath{n}\xspace}
\newcommand{\witruntime}{\ensuremath{\witsize\log\witsize}\xspace}
\newcommand{\tabsize}{\ensuremath{N}\xspace}
\newcommand{\tabruntime}{\ensuremath{\tabsize\log\tabsize}\xspace}
\newcommand{\tab}{\ensuremath{\mathfrak{t}}\xspace}
 \renewcommand{\a}{\ensuremath{\mathsf{a}}\xspace}
\renewcommand{\b}{\ensuremath{\mathsf{b}}\xspace}
\renewcommand{\c}{\ensuremath{\mathsf{c}}\xspace}
\newcommand{\f}{\ensuremath{\mathsf{f}}\xspace}
\renewcommand{\p}{\ensuremath{\mathsf{p}}\xspace}
\newcommand{\qa}{\ensuremath{\mathsf{q_a}}\xspace}
\newcommand{\qb}{\ensuremath{\mathsf{q_b}}\xspace}
\newcommand{\m}{\ensuremath{\mathsf{m}}\xspace}
\newcommand{\agam}{\ensuremath{a_\gamma}\xspace}
\newcommand{\gamproof}{\ensuremath{\mathsf{\pi_\gamma}}\xspace}
\newcommand{\zerproof}{\ensuremath{\mathsf{\pi_0}}\xspace}
\newcommand{\bgam}{\ensuremath{b_\gamma}\xspace}
\newcommand{\qbgam}{\ensuremath{Q_{b,\gamma}}\xspace}
\newcommand{\zgam}{\ensuremath{Z_{\bigspace,\gamma}}\xspace}
\newcommand{\fgam}{\ensuremath{f_{\gamma}}\xspace}
\newcommand{\pgam}{\ensuremath{P_{\gamma}}\xspace}

\begin{document}
    \maketitle
\begin{abstract}
 We present a protocol for checking the values of a committed polynomial $\phi(X)\in \polysofdeg{\tabsize}$ over a multiplicative subgroup $\subspace\subset \F$ of size \witsize are contained in a
 table $T\in \F^\tabsize$. After an $O(\tabsize \log \tabsize)$ preprocessing step, the prover algorithm runs in time $O(\witsize\log \witsize)$. Thus, we continue to improve upon the recent breakthrough sequence of results\cite{caulk,caulkp,flookup,baloo} starting from \caulk~\cite{caulk}, which achieve sublinear complexity in the table size \tabsize. The two most recent works in this seqence \cite{flookup,baloo} achieved
 prover complexity $O(\witsize\cdot \log^2 \witsize)$.
 
 Moreover, as in \cite{caulk,caulkp,baloo} our construction relies on homomorphic table commitments, which makes them amenable to vector lookups in the manner described in Section 4 of \cite{plookup}.
 \end{abstract}

% !TEX root = ../flookup.texV

\section{Introduction}
The \emph{lookup problem} is fundamental to the efficiency of modern zk-SNARKs.
Somewhat informally, it asks for a protocol to prove the values of a committed polynomial $\phi(X)\in\polysofdeg{\witsize}$ are contained in a table $T$ of size $\tabsize$ of predefined legal values.
When the table $T$ corresponds to an operation without an efficient low-degree arithmetization in $\F$, such a protocol produces significant savings in proof construction time for programs containing the operation.
Building on previous work of \cite{arya}, \plookup \cite{plookup} was the first to explicitly describe a solution to this problem in the polynomial-IOP context. \plookup described a protocol with prover complexity quasilinear in both \witsize and \tabsize.
This left the intriguing question of whether the dependence on \tabsize could be made \emph{sublinear} after performing a preprocessing step for the table $T$.
\caulk \cite{caulk} answered this question in the affirmative by leveraging bi-linear pairings, achieving a run time of $O(\witsize^2+\witsize\log \tabsize)$. \caulkp \cite{caulkp} improved this to $O(\witsize^2)$ getting rid of the dependence on table size completely. 

However, the quadratic dependence on \witsize  of these works makes them impractical for a circuit with many lookup gates.
We resolve this issue by giving a protocol called \cq that is quasi-linear in \witsize and has no dependence on \tabsize after the preprocessing step.

\subsection{Comparison of results}

Table with relative proof size, prover ops, verifier ops
                proof-size
caulk
caulk+
flookup
baloo         12 \G1, 1 \G2, 4 \F
this work  6 G1, 1 G2

\subsection{Technical Overview}
\paragraph{The innovation of \caulk }
While \cite{caulk,caulkp,flookup,baloo} use preprocessing and pairings to extract a subtable of witness size;


\paragraph{Our approach}
here we use preprocessing and pairings more directly to run an existing lookup protocol - mvlookup,
in time independent from table size
-logarithmic derivative method
Let's review this protocol:
It relies on the following lemma from \cite{mvlookup}
that says that $\restricttoset{f}{\subspace}\in \tab$ if and only if for some $m\in \F^\tabsize$
 \[\sumi{\tabsize}\frac{m_i}{X+t_i}=\sumi{\witsize}\frac{1}{X+f_i}\]
Roughly, the protocol of \cite{mvlookup} checks this identity on a random $\beta$,
by sending polynomials $A$ and $B$ that agree on \bigspace with the rational function values of the LHS and RHS
respectively.
Given commitments to $A,B$ we can check the equality holds via various sumcheck techniques, e.g. that descirbed in \cite{aurora}.
The RHS is not a problem because it is a sum of size \witsize.
Interpolating $A$, and computing its commitment is actually not a problem either, because the number of non-zero values
is at most \witsize. So if we precompute the commitments to the Lagrange base of  \bigspace we're fine.

The main challenge, and innovation, is to convince the verifier \ver that $A$ is correctly formed.


This protocol is amenable, because polynomials involved have sparsity depending on witness
- For large table problem is computing A that agrees with $m/(\tab+\beta)$ on \bigspace

- Need way to compute $A$ 
\section{Preliminaries}

\subsection{Notation:}
\subspace - small space
\bigspace - big space
Lagrange bases for big and small space




AGM - real and ideal pairing checks,
agm - real and ideal pairing 
KZG

\subsection{log derivative method}
Lemma from mvlookup
\begin{lemma}\label{lem:mvlookup}
 Given $f\in \F^\witsize$, and $t\in \F^\tabsize$,
 we have $f\subset t$ as sets if and only if for some $m\in \F^\tabsize$ the following identity of rational functions holds
 \[\sumi{\witsize}\frac{1}{X+f_i}=\sumi{\tabsize}\frac{m_i}{X+t_i}.\]
\end{lemma}

\section{Cached quotients}

\begin{thm}
Fix $T\in \polysofdeg{\tabsize}$, and a subgroup $\bigspace\subset \F$ of size \tabsize. 
There is an algorithm that after a preprocessing step of $O(\tabsize\cdot \log \tabsize)$ operations.
Given input $f\in \polysofdeg{\witsize}$ computes in $O(\witsize\cdot \log \witsize)$ \G2 operations
$\cm=\enc2{Q(x)}$ where $Q\in \polysofdeg{\tabsize}$ is such that
\[f(X)\cdot T(X) = Q(X)\cdot Z_{\bigspace}(X) + R(X),\]
for $R(X)\in \polysofdeg{\tabsize}$
\end{thm}


\begin{lemma}
Fix $T\in \polysofdeg{\tabsize}$, and a subgroup $\bigspace\subset \F$ of size \tabsize. 
There is an algorithm that given the \G1 elements $\sett{\enc1{x^i}}{i\in \set{0,\ldots,\tabsize}}$ computes for $i\in [\tabsize]$, the elements 
$q_i\defeq \enc1{Q_i(x)}$
where $Q_i(X)\in \F[X]$ is such that
\[L_i(X)\cdot T(X)=t_i\cdot L_i(X) + Z_\bigspace(X)\cdot Q_i(X)\]
in $O(\tabsize\cdot \log \tabsize)$ \G1 operations.
 
\end{lemma}

\begin{lemma}
Fix $T\in \polysofdeg{\tabsize}$, and a subgroup $\bigspace\subset \F$ of size \tabsize. 
There is an algorithm that given the \G1 elements $\sett{\enc1{x^i}}{i\in \set{0,\ldots,d}}$ computes for $i\in [\tabsize]$, the elements 
$q_i\defeq \enc1{^{d-\tabsize}\cdot Q_i(x)}$
where $Q_i(X)\in \F[X]$ is such that
\[L_i(X)\cdot T(X)=t_i\cdot L_i(X) + Z_\bigspace(X)\cdot Q_i(X)\]
in $O(\tabsize\cdot \log \tabsize)$ \G1 operations.
 
\end{lemma}

\section{Main protocol}
\begin{dfn}\label{dfn:mainrelation}
 \rel is all pairs $(\cm,f)$
 such that \cm is a commitment to $f$ and $f|_\subspace \subset T$.
 
 ..bla  problem is relation is defined only after srs is chosen
\end{dfn}

\subsection{Definitions}

Ad-hoc dfn of ks protocol for table lookup

Relations dependent on srs.
Tuple \gen,\isintable
\begin{itemize}
 \item $\gen(\tab,\tabsize)\to \srs0$
 \item $\isintable$ a protocol between \prv and \ver where \prv has input $f\in \polysofdeg{\witsize}$, \ver has \enc1{f(x)}.
 Both have \tab and \srs0.
 such that
 \begin{itemize}
  \item Completeness:If $\restricttoset{f}{\subspace}\subset \tab$ then \ver outputs \acc with probability one.
  \item Knowledge soundness in the algebraic group model:
   For any $\tab\in\F^\witsize$, the probability of any algebraic \adv to win the following game is \negl

\begin{enumerate}
 \item Let $\srs0=\gen(\tab,\tabsize)$.
 \item \adv sends a message \cm and values $f_1,\ldots, f_\witsize$ such that $\cm=\sumi{\witsize}f_i\cdot \enc1{L_i(x)}$.
 \item \adv and \ver engage in the protocol $\isintable(\tab,\cm)$ with \adv taking the role of \prv.
 \item \adv wins if 
 \begin{itemize}
  \item \ver outputs \acc
  \item $\restricttoset{f}{\subspace}\not\subset \tab$.
 \end{itemize}

\end{enumerate}
\end{itemize}
\end{itemize}
Main protocol:
Preprocessed inputs:
\enc2{Z_\bigspace(x)}, \enc2{T(x)}
Input $(\cm,f)$.
	\paragraph{\textbf{Round 1:Committing to the multiplicites vector}} \ \\
\begin{enumerate}
 \item \prv computes poly $m\in \polysofdeg{\tabsize}$ such that $m_i=$ number of times $\tab_i$ appears in \restricttoset{f}{\subspace}
 \item \prv sends $\m\defeq \enc1{m(x)}$.
 \end{enumerate}
	\paragraph{\textbf{Round 2:Interpolating the rational identity at a random $\beta$; checking the identity for $A$ using pairings}} \ \\
 
\begin{enumerate}
 \item \ver chooses and sends random $\beta \in \F$.
\item \prv computes $A\in \polysofdeg{\tabsize}$ such that for $i\in [\tabsize]$, $A_i = m_i/(\tab_i+\beta)$.
\item \prv sends $\a\defeq \enc1{A(x)}$.
\item\label{step:computeQA} \prv computes $\qa \defeq \enc2{Q_A(x)}$ where $Q_A\in \polysofdeg{\tabsize}$ is such that 
\[A(X)(T(X)+\beta)-m(X)= Q_A(X)\cdot Z_\bigspace(X)\]
\item \prv computes $B\in \polysofdeg{\witsize}$ such that for $i\in [\witsize]$, $B_i=1/(f_i+\beta)$.
\item \prv sends $\qb\defeq \enc1{B(x)}$.
\item \prv computes $Q_B(X)$ such that 
\[B(X)(f(x)+\beta)-1 = Q_B(X)\cdot Z_\subspace(X)\]
\item \prv computes and sends the value $a_0\defeq A(0)$.
\item \ver  sets $b_0\defeq (\tabsize\cdot a_0)/\witsize$.
\item \prv computes and sends $\p=\enc1{P(x)}$ where
\[P(X)\defeq A(X)\cdot X^{d-\tabsize} \]
\item\label{step:checkqa} \ver checks that $A$ encodes the correct values:
\[e(\a,\enc2{T(x)}+\enc2{\beta} )=e(\qa,\enc2{Z_\bigspace(x)})\cdot e(\m,\enc2{1})\]
\item\label{step:checkadeg} \ver checks that $A$ has the appropriate degree:
\[e(\a,\enc2{x^{d-\tabsize}})=e(\p,\enc2{1}).\]
 \end{enumerate}
	\paragraph{\textbf{Round 3: Checking the identities for $B$ at random $\gamma \in \F$}}
 \begin{enumerate}
\item \ver sends random $\gamma,\eta,\zeta \in \F$.
\item \prv sends $\bgam \defeq B(\gamma),\qbgam \defeq Q_B(\gamma),\fgam \defeq f(\gamma)$.
\item As part of checking the correctness of \qb, \ver computes $ Z_\subspace(\gamma) = \gamma^\witsize -1$ and computes
\[\qbgam\defeq \frac{\bgam\cdot (\fgam + \beta)-1}{Z_\subspace(\gamma)}\]
\item As part of  checking $P$ is correct, \ver computes 
\[\pgam\defeq \bgam\cdot \gamma^{d-\witsize}\]
\item To perform a batched KZG check for the correctness of the values  $\agam,\bgam,\fgam,\pgam$
  \begin{enumerate}
   \item \ver sends random $\eta\in \F$. \prv and \ver separately compute
   \[v\defeq \bgam +\eta\cdot\fgam + \eta^2\cdot \qbgam + \eta^3\cdot \pgam\]
   \item \prv computes $\gamproof\defeq \enc1{h(x)}$ for 
   \[h(X)\defeq \frac{ B(X)+ \eta\cdot f(X) + \eta^2\cdot Q_B(X) + \eta^3\cdot P(X) -v}{X-\gamma}\]
   \item \ver computes 
   \[\c\defeq \b+\eta\cdot \f +\eta^2\cdot \qb + \eta^3\cdot \p\]
   and  checks that 
   \[e(\c-\enc1{v}+\gamma\cdot \gamproof,\enc2{1})=e(\gamproof,\enc2{x})\]
  \end{enumerate}
\item To perform a batched KZG check for the correctness of the values  $a_0,b_0$
  \begin{enumerate}
   \item  \prv and \ver separately compute
   \[u\defeq a_0+\zeta \cdot b_0.\] 
   \item \prv computes and sends $\zerproof\defeq \enc1{h_0(x)}$ for 
   \[h_0(X)\defeq \frac{A(X)+\zeta\cdot B(X)}{X}\]
   \item \ver computes 
   \[\c_0 \defeq \a +\zeta \b\]
   and  checks that 
   \[e(\c_0-\enc1{u},\enc2{1})=e(\zerproof,\enc2{x})\]
  \end{enumerate}

\end{enumerate}

\paragraph{Stats:}

verifier pairings:4 - pair \a with random combination of $T$ and \enc2{x^{d-\tabsize}}, pair \qa with $Z_\bigspace$.

\begin{lemma}\label{lem:computeQA}
The element $q_A$ in Step \ref{step:computeQA} can be computed in  \witruntime \G2-operations and $O(\witruntime)$ \F-operations
\end{lemma}
\begin{lemma}\label{lem:computeKZGproofs}
The elements $\zerproof,\gamproof$ can be computed in  $2\cdot\witruntime$ \G1-operations and $O(\witruntime)$ \F-operations
\end{lemma}

\paragraph{Knowledge soundness proof:}

Look at the following events 
\bibliographystyle{alpha}
\bibliography{references}
\end{document}


